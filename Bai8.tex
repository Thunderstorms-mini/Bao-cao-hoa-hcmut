\chapter{Bài 8: PHÂN TÍCH THỂ TÍCH}

\section{Kết quả thí nghiệm}

\subsection{Thí nghiệm 1: Xác định đường cong chuẩn độ HCl bằng NaOH}
\begin{center}
    \begin{figure}[htp]
        \begin{center}
            \includegraphics[scale=1]{Dothi.png}
        \end{center}
    \end{figure}
\end{center}
Xác định:
\begin{itemize}
    \item pH điểm tương đương.
    \item Bước nhảy pH: từ pH ..... đến pH .....
\end{itemize}

\newpage

\subsection{Thí nghiệm 2: Tìm nồng độ HCl với phenolphtalein}
\begin{center}
\begin{tabular}{|c|c|c|c|c|c|}
    \hline \textbf{Lần} & $V_{HCl}$ (ml) & $V_{NaOH}$ (ml) & $C_{NaOH}$ (N) & $C_{HCl}$ (N) & Sai số\\
    \hline       1      &   10           &                 &      0,1       &               &       \\
    \hline       2      &   10           &                 &      0,1        &               &       \\
    \hline
\end{tabular}
\end{center}
$\bar C_{HCl}$=...; N=.....

\subsection{Thí nghiệm 3: Tìm nồng độ HCl với Metyl da cam}
\begin{center}
    \begin{tabular}{|c|c|c|c|c|c|}
        \hline \textbf{Lần} & $V_{HCl}$ (ml) & $V_{NaOH}$ (ml) & $C_{NaOH}$ (N) & $C_{HCl}$ (N) & Sai số\\
        \hline       1      &   10           &                 &      0,1       &               &       \\
        \hline       2      &   10           &                 &      0,1        &               &       \\
        \hline
    \end{tabular}
\end{center}

\subsection{Thí nghiệm 4: Tìm nồng độ $CH_{3}COOH$}
\begin{center}
    \begin{tabular}{|c|c|c|c|c|c|}
        \hline \textbf{Lần} & \textbf{Chất chỉ thị} & $V_{CH_3COOH}$ (ml) & $V_{NaOH}$ (ml) & $C_{NaOH}$ (N) & $C_{CH_3COOH}$ (N)\\
        \hline      1       &  Phenol phtalein      &          10         &                 &     0,1        &                   \\
        \hline      2       &  Metyl orange         &          10         &                 &     0,1        &                    \\
        \hline
    \end{tabular}
\end{center}


\section{Trả lời câu hỏi}

\textbf{1.} Khi thay đổi nồng độ HCl và NaOH, đường cong chuẩn độ có thay đổi không, tại sao? \\

\textbf{2.} Việc xác định nồng độ axit HCl trong các thí nghiệm 2 và 3 cho kết quả chính xác hơn, tại sao?\\

\textbf{3.} Từ kết quả thí nghiệm 4, việc xác định nồng độ dung dịch axit acetic bằng chỉ thị màu nào chính xác hơn, tại sao?\\

\textbf{4.} Trong phép phân tích thể tích, nếu đổi vị trí của NaOH và axit thì kết quả có thay đổi không, tại sao?\\


\begin{center}
    \textbf{(HẾT PHẦN BÁO CÁO)}
\end{center}
\newcommand{\xfill}[2][1ex]{{%
  \dimen0=#2\advance\dimen0 by #1
  \leaders\hrule height \dimen0 depth -#1\hfill%
}}

\ \xfill{1pt} \

* PHÂN CÔNG BÀI LÀM *
