\chapter{BÀI 2: NHIỆT PHẢN ỨNG}

   \section{Kết quả thí nghiệm}
   
      \subsection{Thí nghiệm 1: Tìm \textbf{$m_{0}c_{0}$}}
      
         \begin{center}
            \begin{tabular}{|c|c|c|c|}
               \hline \textbf{Nhiệt độ ($ ^{o}C$)}&\textbf{Lần 1}&\textbf{Lần 2}&\textbf{Lần 3}\\
               \hline \textbf{$t_{1}$}&33&&\\
               \hline \textbf{$t_{2}$}&56&&\\
               \hline \textbf{$t_{3}$}&44&&\\
               \hline \textbf{$m_{0}C_{0}$}&4,16&&\\
               \hline
            \end{tabular}
         \end{center}
         - Với $m = 50g$ và $c = 1$ cal/độ.\\
         - $m_{0}C_{0} = mc\frac{(t_{3}-t_{1})-(t_{2}-t{3})}{t_{2}-t_{3}} = 4,16$ cal/độ. 
         
      \subsection{Thí nghiệm 2: Nhiệt phản ứng HCl và NaOH}
      
         \begin{center}
             \begin{tabular}{|c|c|c|c|}
                  \hline \textbf{Nhiệt độ ($ ^{o}C$)}&\textbf{Lần 1}&\textbf{Lần
                  2}&\textbf{Lần 3}\\
                  \hline \textbf{$t_{1}$}&33&&\\
                  \hline \textbf{$t_{2}$}&32&&\\
                  \hline \textbf{$t_{3}$}&38&&\\
                  \hline Q(cal)&&&\\
                  \hline $Q_{tb}$(cal)& \multicolumn{3}{c|}{}\\
                  \hline $\Delta$H(cal/mol)&\multicolumn{3}{c|}{}\\
                  \hline
             \end{tabular}
         \end{center} 
         - Với m = 50g; c = 1 cal/độ; $\Delta n$ = 0,025 mol; $m_{o}c_{o}$ = 4,525 cal/độ.\\
         - $\Delta t = t_{3} - \frac{t_{1}+t_{2}}{2} =  ^{o}C$.\\
         - $Q = (m_{o}c_{o} + mc).\Delta t=  = cal$.\\
         - $\Delta H = \frac{-Q}{\Delta n} =  cal$.\\
         - Vậy: $Q = cal$; $\Delta H = cal$.\\
         
      \newpage
      
      \subsection{Thí nghiệm 3: Nhiệt độ hòa tan $CuSO_{4}$}
      
         \begin{center}
             \begin{tabular}{|c|c|c|c|}
                  \hline Nhiệt độ ( $^{o}C$)&\textbf{Lần 1}&\textbf{Lần 2}&\textbf{Lần 3}\\
                  \hline $t_{1}$&35&&\\
                  \hline $t_{2}$&39&&\\
                  \hline Q(cal)&&&\\
                  \hline $\Delta H$ (cal/mol)&&&\\
                  \hline $\Delta H_{tb}$ (cal/mol)&\multicolumn{3}{c|}{}\\
                  \hline
             \end{tabular}
         \end{center}

      \subsection{Thí nghiệm 4: Nhiệt hòa tan $NH_{4}Cl$}
      
         \begin{center}
             \begin{tabular}{|c|c|c|c|}
                  \hline Nhiệt độ ( $^{o}C$)&\textbf{Lần 1}&\textbf{Lần 2}&\textbf{Lần 3}\\
                  \hline $t_{1}$&37&&\\
                  \hline $t_{2}$&31&&\\
                  \hline Q(cal)&&&\\
                  \hline $\Delta H$ (cal/mol)&&&\\
                  \hline $\Delta H_{tb}$ (cal/mol)&\multicolumn{3}{c|}{}\\
                  \hline
             \end{tabular}
         \end{center}
         
   \section{Trả lời câu hỏi}
   
   \textbf{1.} $\Delta H_{th}$ của phản ứng $HCl + NaOH \to NaCl + H_{2}O$ sẽ được tính theo số mol Hcl hay NaOH khi cho 25ml dung dịch HCl 2M tác dụng với 25ml dung dịch NaOH 1M? Tại sao?\\
   
   \hspace{2cm} $HCl   +   NaOH \to NaCl + H_{2}O$\\
     Ban đầu:\hspace{1.1cm}0,05 \hspace{0.4cm}0.025 \hspace{3.8cm}(mol)\\
     Phản ứng: \hspace{0.75cm}0,025 \hspace{0.2cm}0,025 \hspace{3.8cm} (mol)\\
     Còn lại:\hspace{1.25cm}0.025 \hspace{0.47cm}0 \hspace{4.25cm} (mol)\\
   
   - Ta thấy NaOH hết và HCl còn dư, nên $\Delta H_{th}$ của phản ứng tính theo $NaOH$. Vì lượng HCl dư không tham gia phản ứng nên không sinh nhiệt.\\
   
   \textbf{2.} Nếu thay thế HCl 1M bằng $HNO_{3}$ 1M thì kết quả thí nghiệm 2 có thay đổi không?\\
   
   - Kết quả vẫn không thay đổi, vì $\Delta H$ là đại lượng đặc trưng cho mỗi phản ứng, mà sau khi thay đổi HCl bằng $HNO_{3}$ thì vẫn là phản ứng trung hòa: $HNO_{3} + NaOH \to NaNO_{3} + H_{2}O$. \\
  
  \newpage
  
  \textbf{3.} Tính $\Delta H_{3}$ bằng lý thuyết theo định luật Hess. So sánh với kết quả thí nghiệm . Hãy xem 6 nguyên nhân có thể gây ra sai số trong thí nghiệm này:\\
   + Mất nhiệt do nhiệt lượng kế.\\
   + Do nhiệt kế.\\
   + Do dụng cụ đong thể tích hóa chất.\\
   + Do cân.\\
   + Do sunfat đồng bị hút ẩm.\\
   + Do lấy nhiệt dung riêng dung dịch sunfat đồng bằng 1 cal/mol.độ.\\
   Theo em, sai số nào là quan trọng nhất? Còn nguyên nhân nào khác không?\\
   
   \noindent - Theo em, kết quả thí nghiệm nhỏ hơn so với lý thuyết.\\
   - Nguyên nhân quan trọng nhất gây ra sai số sunfat đồng bị hút ẩm:\\ $CuSO_{4_khan} + 5H_{2}O \to CuSO_{4}.5H_{2}O$ tạo ra $\Delta H_{1}$ nữa,hoặc do ở dạng ngậm nước nên tạo ra nhiệt lượng thấp hơn so với lý thuyết. Mặt khác, $CuSO_{4}$ hút ẩm thì số mol sẽ khác so với tính toán trên lý thuyết ($CuSO_{4}$ khan). 